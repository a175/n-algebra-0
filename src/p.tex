% !TeX root =./x2.tex
% !TeX program = pdfpLaTeX
\chapter{演算}
\section{演算について}
集合$X$と集合$Y$が与えられているとする.
以下の条件を満たすとき$f$を演算と呼ぶ:
\begin{enumerate}
 \item 
 \label{well-def-map:1}
 $X$の各元$x$に対して, $f$による像$f(x)$が定まっている.
 \item 
 \label{well-def-map:2}
 どの$x\in X$の像$f(x)$も$Y$の元である.
 \item 
\label{well-def-map:3}
各$x\in X$に対し$f(x)$はただ一つ定まっていなければならない.
つまり, $x,x'\in X$に対し,
 \begin{align*}
  x=x'\implies f(x)=f(x').
 \end{align*}
\end{enumerate}
このように定義される演算$f$は,
集合$X$から集合$Y$への写像
そのものである.
つまり, 演算の定義は以下のように言い換えることができる.
\begin{definition}
集合$X$から集合$Y$への
写像$f$のことを演算と呼ぶ.
集合$X$として, $X_1\times \cdots \times X_n$という$n$個の集合の直積を考える場合には,
$n$項演算と呼ぶ.
$n=1$のときには単項演算と呼ぶ.
\end{definition}
$X$の元に$Y$を対応させる`対応'としての側面を意識しているとき
`写像'という述語を用いることが多く,
$X$の元から$Y$を作る`操作'としての側面を意識しているとき
`演算'という述語を用いることが多いように思うが,
数学的には差はない.

等式が与えられたときに,
その両辺に同じ演算を行っても等式は保たれる
という原則が,
演算に求められている条件\cref{well-def-map:3}により導かれる.
この原則は,
これから計算をする上で,
頻繁にかつ無意識に使われる重要なものである.
\cref{well-def-map:3}は一見いつでも満たされるように思うかもしれないが,
表記は異なるが等しくなる可能性がある元を$X$が含むときには,
\cref{well-def-map:3}が満たされない可能性がある.
新しく演算を定義する際には,
注意する必要がある.

以下で, 単項演算と二項演算の例をいくつか挙げる.
\begin{example}
  実数$\theta$から,
  各$\theta$に対する正弦$\sin(\theta)$を得ることを考える.
  \begin{align*}
    \shazo{\sin}
          {\RR}{\RR}
          {x}{\sin(x)}
  \end{align*}
  という写像を考えると言ってもいい.
  このとき, $\sin$は単項演算である.  
\end{example}
これらの例や定義では, 演算を表す記号を,
関数を表す際に通常行われるように,
先頭に書いているが,
このような表記を用いる必要はない.
\begin{example}
  正の実数$x$に対し, その逆数を対応させる
  \begin{align*}
    \shazo{\bullet^{-1}}
          {\RR}{\RR}
          {x}{\frac{1}{x}}
  \end{align*}
  も単項演算である.
  ${}^{-1}$だけでは, どこに実数$x$を代入すればよいかわかりにくいため,
  代入する部分をはっきりさせるため,
  $\bullet^{-1}$とここでは書いている.
  このように, 先頭以外に演算を表す記号を書くこともある.
\end{example}
\begin{example}
  実数$x$に対し, その絶対値を対応させる
  \begin{align*}
    \shazo{|\bullet|}
          {\RR}{\RR}
          {x}{\begin{cases}x&(x\geq 0)\\-x&(x<0)\end{cases}}
  \end{align*}
  も単項演算である.
  このように, `カッコ'を使って演算を表すこともある.
\end{example}
\begin{example}
  非負整数$x$に対し, その階乗を対応させる
  \begin{align*}
    \shazo{\bullet !}
          {\NN}{\NN}
          {x}{x(x-1)(x-2)\cdots 1}
  \end{align*}
  も単項演算である.
  このように, 演算を表す記号を後ろに書くような場合もある.
\end{example}
\begin{example}
  複素数$x$に対し, その複素共軛を対応させる
  \begin{align*}
    \shazo{\bar\bullet }
          {\CC}{\CC}
          {x}{\bar{x}}
  \end{align*}
  も単項演算である.
\end{example}
\begin{example}
  非負実数$x$に対し, その平方根を対応させる
  \begin{align*}
    \shazo{\sqrt\bullet }
          {\RR_{>0}}{\RR}
          {x}{\sqrt{x}}
  \end{align*}
  も単項演算である.
\end{example}
\begin{example}
  整数$x,y$に対し, その和を対応させる
  \begin{align*}
    \shazo{\bullet+\bullet}
          {\ZZ\times \ZZ}{\ZZ}
          {(x,y)}{x+y}
  \end{align*}
  は二項演算である.
  二項演算では, 演算を表す記号を中央に書く場合もある.
\end{example}
\begin{example}
  実ベクトル$x,y$に対し, その内積を対応させる
  \begin{align*}
    \shazo{\Braket{\bullet,\bullet}}
          {\RR^n\times \RR^n}{\RR}
          {(x,y)}{\Braket{x,y}}
  \end{align*}
  も二項演算である.
  二項演算でも, `カッコ'を使って演算を表すこともある.
\end{example}

写像と演算の組を代数系と呼ぶ.
演算が複数与えられることもある.
本稿の目的は,
代表的な代数系について紹介することにある.

\section{単項演算}
集合$X$から集合$Y$への写像を単項演算と呼んだ.
特に, $X$から$X$への写像を$X$上の単項演算と呼ぶ.
例えば, $X$上の恒等写像$\id_X$は, $X$上の単項演算である.
\begin{definition}
  次の条件を満たす$X$上の演算$f$をinvolutionと呼ぶ:
  全ての$x\in X$に対し,
  \begin{align*}
    f(f(x))=x.
  \end{align*}
\end{definition}
\begin{example}
  実数$x$に対し, その$-1$倍を対応させる
  \begin{align*}
    \shazo{-\bullet}
          {\RR}{\RR}
          {x}{-x}
  \end{align*}
  は$\RR$上の単項演算である.
  $-(-x)=x$であるので,
  involutionである.
\end{example}
\begin{example}
  正の実数$x$に対し, その逆数を対応させる
  \begin{align*}
    \shazo{\bullet^{-1}}
          {\RR_{>0}}{\RR_{>0}}
          {x}{x^{-1}}
  \end{align*}
  は$\RR_{>0}$上の単項演算である.
  $(x^{-1})^{-1}=x$であるので,
  involutionである.
\end{example}
\begin{example}
  正の実数$x$に対し, その平方根を対応させる
  \begin{align*}
    \shazo{\sqrt{\bullet}}
          {\RR_{>0}}{\RR_{>0}}
          {x}{\sqrt{x}}
  \end{align*}
  は$\RR_{>0}$上の単項演算である.
  $16\in\RR_{>0}$に対し,
  $\sqrt{\sqrt{16}}=\sqrt{4}=2\neq 16$であるので,
  involutionではない.
\end{example}
\begin{prop}
  $f$が$X$上のinvolutionであるとき,
  \begin{align*}
    \shazo{\varphi}
          {X}{X}
          {x}{f(x)}
  \end{align*}
  は全単射である,
\end{prop}
\begin{proof}
  $f$がinvolutionであるので, $\varphi\circ\varphi=\id_X$である.
  これは, $\varphi$が$\varphi$自身の逆写像であることを意味する.
\end{proof}

\section{二項演算}
集合$X\times X'$から集合$Y$への写像を二項演算と呼んだ.
特に, $X\times X$から$X$への写像を$X$上の二項演算と呼ぶ.
また, $X\times Y$から$Y$への写像を$X$の$Y$への左作用と呼び,
$Y\times X$から$Y$への写像を$X$の$Y$への右作用と呼ぶ.
$X$から$Y$への作用を考える際には,
$X$や$Y$が代数系であることが多く,
その場合にはそれらの演算と作用の整合性に関する条件が要請されることが多い.
また, $X\times X$から$Y$への写像を$X$上の$Y$-形式と呼ぶことがある.
この場合にもそれらの演算と作用の整合性に関する条件が要請されることが多い.

\subsection{$X$上の二項演算}
\sectionX{章末問題}
\begin{enumerate}
  \item
\end{enumerate}


