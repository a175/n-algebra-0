% !TeX root =./x2.tex
% !TeX program = pdfpLaTeX
\chapter{演算}
\section{演算について}
集合$X$と集合$Y$が与えられているとする.
以下の条件を満たすとき$f$を演算と呼ぶ:
\begin{enumerate}
 \item 
 \label{well-def-map:1}
 $X$の各元$x$に対して, $f$による像$f(x)$が定まっている.
 \item 
 \label{well-def-map:2}
 どの$x\in X$の像$f(x)$も$Y$の元である.
 \item 
\label{well-def-map:3}
各$x\in X$に対し$f(x)$はただ一つ定まっていなければならない.
       つまり, $x,x'\in X$に対し,
 \begin{align*}
  x=x'\implies f(x)=f(x').
 \end{align*}
\end{enumerate}
このように定義される演算$f$は,
集合$X$から集合$Y$への写像
そのものである.
つまり, 演算の定義は以下のように言い換えることができる.
\begin{definition}
集合$X$から集合$Y$への
写像$f$のことを演算と呼ぶ.
\end{definition}
$X$の元に$Y$を対応させる`対応'としての側面を意識しているとき
`写像'という述語を用いることが多く,
$X$の元から$Y$を作る`操作'としての側面を意識しているとき
`演算'という述語を用いることが多いように思うが,
数学的には差はない.

等式が与えられたときに,
その両辺に同じ演算を行っても等式は保たれる
という原則が,
演算に求められている条件\cref{well-def-map:3}により導かれる.
この原則は,
これから計算をする上で,
頻繁にかつ無意識に使われる重要なものである.
\cref{well-def-map:3}は一見いつでも満たされるように思うかもしれないが,
表記は異なるが等しくなる可能性がある元を$X$が含むときには,
満たされない可能性がある.
新しく演算を定義する際には,
注意する必要がある.

演算の例をいくつか挙げる:
前置記法単項演算.
前置記法2項演算.
中置記法2項演算.
内積型.

\sectionX{章末問題}
\begin{enumerate}
  \item
\end{enumerate}

