% !TeX root =./x2.tex
% !TeX program = pdfpLaTeX
\chapter{演算}
\section{演算について}
集合$X$と集合$Y$が与えられているとする.
以下の条件を満たすとき$f$を演算と呼ぶ:
\begin{enumerate}
 \item 
 \label{well-def-map:1}
 $X$の各元$x$に対して, $f$による像$f(x)$が定まっている.
 \item 
 \label{well-def-map:2}
 どの$x\in X$の像$f(x)$も$Y$の元である.
 \item 
\label{well-def-map:3}
各$x\in X$に対し$f(x)$はただ一つ定まっていなければならない.
つまり, $x,x'\in X$に対し,
 \begin{align*}
  x=x'\implies f(x)=f(x').
 \end{align*}
\end{enumerate}
このように定義される演算$f$は,
集合$X$から集合$Y$への写像
そのものである.
つまり, 演算の定義は以下のように言い換えることができる.
\begin{definition}
集合$X$から集合$Y$への
写像$f$のことを演算と呼ぶ.
集合$X$として, $X_1\times \cdots \times X_n$という$n$個の集合の直積を考える場合には,
$n$項演算と呼ぶ.
$n=1$のときには単項演算と呼ぶ.
\end{definition}
$X$の元に$Y$を対応させる`対応'としての側面を意識しているとき
`写像'という述語を用いることが多く,
$X$の元から$Y$を作る`操作'としての側面を意識しているとき
`演算'という述語を用いることが多いように思うが,
数学的には差はない.

等式が与えられたときに,
その両辺に同じ演算を行っても等式は保たれる
という原則が,
演算に求められている条件\cref{well-def-map:3}により導かれる.
この原則は,
これから計算をする上で,
頻繁にかつ無意識に使われる重要なものである.
\cref{well-def-map:3}は一見いつでも満たされるように思うかもしれないが,
表記は異なるが等しくなる可能性がある元を$X$が含むときには,
\cref{well-def-map:3}が満たされない可能性がある.
新しく演算を定義する際には,
注意する必要がある.

以下で, 単項演算と二項演算の例をいくつか挙げる.
\begin{example}
  実数$\theta$から,
  各$\theta$に対する正弦$\sin(\theta)$を得ることを考える.
  \begin{align*}
    \shazo{\sin}
          {\RR}{\RR}
          {x}{\sin(x)}
  \end{align*}
  という写像を考えると言ってもいい.
  このとき, $\sin$は単項演算である.  
\end{example}
これらの例や定義では, 演算を表す記号を,
関数を表す際に通常行われるように,
先頭に書いているが,
このような表記を用いる必要はない.
\begin{example}
  正の実数$x$に対し, その逆数を対応させる
  \begin{align*}
    \shazo{\bullet^{-1}}
          {\RR}{\RR}
          {x}{\frac{1}{x}}
  \end{align*}
  も単項演算である.
  ${}^{-1}$だけでは, どこに実数$x$を代入すればよいかわかりにくいため,
  代入する部分をはっきりさせるため,
  $\bullet^{-1}$とここでは書いている.
  このように, 先頭以外に演算を表す記号を書くこともある.
\end{example}
\begin{example}
  実数$x$に対し, その絶対値を対応させる
  \begin{align*}
    \shazo{|\bullet|}
          {\RR}{\RR}
          {x}{\begin{cases}x&(x\geq 0)\\-x&(x<0)\end{cases}}
  \end{align*}
  も単項演算である.
  このように, `カッコ'を使って演算を表すこともある.
\end{example}
\begin{example}
  非負整数$x$に対し, その階乗を対応させる
  \begin{align*}
    \shazo{\bullet !}
          {\NN}{\NN}
          {x}{x(x-1)(x-2)\cdots 1}
  \end{align*}
  も単項演算である.
  このように, 演算を表す記号を後ろに書くような場合もある.
\end{example}
\begin{example}
  複素数$x$に対し, その複素共軛を対応させる
  \begin{align*}
    \shazo{\bar\bullet }
          {\CC}{\CC}
          {x}{\bar{x}}
  \end{align*}
  も単項演算である.
\end{example}
\begin{example}
  非負実数$x$に対し, その平方根を対応させる
  \begin{align*}
    \shazo{\sqrt\bullet }
          {\RR_{>0}}{\RR}
          {x}{\sqrt{x}}
  \end{align*}
  も単項演算である.
\end{example}
\begin{example}
  整数$x,y$に対し, その和を対応させる
  \begin{align*}
    \shazo{\bullet+\bullet}
          {\ZZ\times \ZZ}{\ZZ}
          {(x,y)}{x+y}
  \end{align*}
  は二項演算である.
\end{example}
\begin{example}
  実ベクトル$x,y$に対し, その内積を対応させる
  \begin{align*}
    \shazo{\Braket{\bullet,\bullet}}
          {\RR^n\times \RR^n}{\RR}
          {(x,y)}{\Braket{x,y}}
  \end{align*}
  も二項演算である.
  二項演算でも, `カッコ'を使って演算を表すこともある.
\end{example}

\begin{example}
  実数$x$と非負整数$n$に対し,
  二項係数を対応させる
  \begin{align*}
    \shazo{\binom{\bullet}{\bullet}}
          {\RR\times \NN}{\RR}
          {(x,n)}{\frac{x(x-1)\cdots(x-n+1)}{n!}}
  \end{align*}
  も二項演算である.
\end{example}

写像と演算の組を代数系と呼ぶ.
演算が複数与えられることもある.
本稿の目的は,
代表的な代数系について紹介することにある.

\section{0項演算}
$n$個の集合の直積$X_1\times \cdots \times X_n$
から$Y$への写像を
$n$項演算と呼んだ.
$0$個の集合の直積は一点集合であると約束し,
一点集合$\Set{\star}$から$Y$への写像を
$0$項演算と呼ぶことがある.
一点集合$\Set{\star}$から$Y$への写像は,
$\star$の像となる$Y$の元を一つ指定することに他ならないから,
これは定数を定めていると思うこともできる.
つまり, $0$項演算とは定数のことであるので,
定数のことを$0$項演算と呼ぶこともある.
定数を与えるという言い回しのほうが平易であるため,
通常は$0$項演算という用語はあまり用いられないように思う.

\section{単項演算}
集合$X$から集合$Y$への写像を単項演算と呼んだ.
特に, $X$から$X$への写像を$X$上の単項演算と呼ぶ.
例えば, $X$上の恒等写像$\id_X$は, $X$上の単項演算である.
\begin{definition}
  次の条件を満たすとき,
  $X$上の単項演算$f$と$g$は可換であるという:
  \begin{enumerate}
  \item $x\in X \implies f(g(x))=g(f(x))$.
  \end{enumerate}
\end{definition}
\begin{example}
  実数を成分とする$n$次正則行列全体からなる集合を$\GL(n,\RR)$とおく.
  $X\in \GL(n,\RR)$に対し, その逆行列を対応させる
  \begin{align*}
    \shazo{\bullet^{-1}}
          {\GL(n,\RR)}{\GL(n,\RR)}
          {X}{X^{-1}}
  \end{align*}
  は$\GL(n,\RR)$上の単項演算である.
  $X\in \GL(n,\RR)$に対し, その転置を対応させる
  \begin{align*}
    \shazo{\transposed{\bullet}}
          {\GL(n,\RR)}{\GL(n,\RR)}
          {X}{\transposed{X}}
  \end{align*}
  は$\GL(n,\RR)$上の単項演算である.
  $X\in \GL(n,\RR)$に対し,
  \begin{align*}
    (\transposed{X})^{-1}=\transposed{(X^{-1})}
  \end{align*}
  であるので, この2つの単項演算は可換である.
\end{example}
\begin{definition}
  次の条件を満たす$X$上の単項演算$f$をinvolutionと呼ぶ:
  全ての$x\in X$に対し,
  \begin{align*}
    f(f(x))=x.
  \end{align*}
\end{definition}
\begin{example}
  実数$x$に対し, その$-1$倍を対応させる
  \begin{align*}
    \shazo{-\bullet}
          {\RR}{\RR}
          {x}{-x}
  \end{align*}
  は$\RR$上の単項演算である.
  $-(-x)=x$であるので,
  involutionである.
\end{example}
\begin{example}
  正の実数$x$に対し, その逆数を対応させる
  \begin{align*}
    \shazo{\bullet^{-1}}
          {\RR_{>0}}{\RR_{>0}}
          {x}{x^{-1}}
  \end{align*}
  は$\RR_{>0}$上の単項演算である.
  $(x^{-1})^{-1}=x$であるので,
  involutionである.
\end{example}
\begin{example}
  正の実数$x$に対し, その平方根を対応させる
  \begin{align*}
    \shazo{\sqrt{\bullet}}
          {\RR_{>0}}{\RR_{>0}}
          {x}{\sqrt{x}}
  \end{align*}
  は$\RR_{>0}$上の単項演算である.
  $16\in\RR_{>0}$に対し,
  $\sqrt{\sqrt{16}}=\sqrt{4}=2\neq 16$であるので,
  involutionではない.
\end{example}
\begin{prop}
  $f$が$X$上のinvolutionであるとき,
  \begin{align*}
    \shazo{\varphi}
          {X}{X}
          {x}{f(x)}
  \end{align*}
  は全単射である.
\end{prop}
\begin{proof}
  $f$がinvolutionであるので, $\varphi\circ\varphi=\id_X$である.
  これは, $\varphi$が$\varphi$自身の逆写像であることを意味する.
\end{proof}
\begin{definition}
  $f$を$X$上の単項演算とし,
  $X'\subset X$とする.
  つぎの条件をみたすとき,
  $X'$は$f$について閉じているという:
  \begin{enumerate}
  \item $x\in X'\implies f(x)\in X'$,
  \end{enumerate}
\end{definition}
\begin{remark}
  $f$を$X$上の単項演算とし,
  $X'\subset X$とする.
  $X'$は$f$について閉じているというのは,
  $f$(の制限)は$X'$上の単項演算でもあるということである,
\end{remark}
\begin{example}
  $\RR^\times = \QQ\setminus\Set{0}$とする.
  このとき,
  $x\in\RR^\times$に対しその逆数を対応させる
  \begin{align*}
    \shazo{\bullet^{-1}}
          {\RR^{\times}}{\RR^{\times}}
          {x}{x^{-1}}
  \end{align*}
  は$\RR^\times$上の単項演算である.

  $x>0$に対し, $x^{-1}>0$であるので,
  $\RR_{>0}$は$\bullet^{-1}$に関して閉じている.
  つまり$\bullet^{-1}$は$\RR_{>0}$上の単項演算でもある.

  $2$に対し, $0<2^{-1}<1$であり整数ではない.
  したがって, $\ZZ_{>0}\subset \RR^{\times}$であるが,
  $\ZZ_{>0}$は$\bullet^{-1}$に関して閉じていない.
\end{example}

$f$と$g$が可換であるとか,
$f$がinvolutionであるといったものは,
演算の性質であり,
その演算に関して閉じた部分集合でも自然にその性質は成り立つ.
\begin{prop}
  $f$と$g$を可換な
  $X$上の単項演算とする.
  $X'\subset X$は, $f$と$g$に関して閉じているとする.
  このとき,
  $X'$上の単項演算として, $f$と$g$は可換である.
\end{prop}
\begin{prop}
  $f$を
  $X$上の単項演算とする.
  $f$はinvolutionであり,
  $X'\subset X$は, $f$に関して閉じているとする.
  このとき,
  $X'$上の単項演算として,
  $f$はinvolutionである.
\end{prop}



\section{二項演算}
集合$X\times X'$から集合$Y$への写像を二項演算と呼んだ.
特に, $X\times X$から$X$への写像を$X$上の二項演算と呼ぶ.
また, $X\times Y$から$Y$への写像を$X$の$Y$への作用と呼ぶ.
$X$の$Y$への作用を考える際には,
$X$や$Y$が代数系であることが多く,
その場合にはそれらの演算と作用の整合性に関する条件が要請されることが多い.
また, $X\times X$から$Y$への写像を$X$上の$Y$-形式と呼ぶことがある.
この場合にもそれらの演算と作用の整合性に関する条件が要請されることが多い.
また, この定義では, $X$上の二項演算は, $X$の$X$への作用でもあり,
$X$上の$X$-形式でもある.

\begin{example}
  実ベクトル$x,y$に対し, その和を対応させる
  \begin{align*}
    \shazo{\bullet+\bullet}
          {\RR^n\times \RR^n}{\RR^n}
          {(x,y)}{x+y}
  \end{align*}
  は$\RR^n$上の二項演算である.
  二項演算では, 演算を表す記号を中央に書く場合もある.
  このような記法を中置記法と呼ぶ.
\end{example}

\begin{example}
 ベクトルのスカラー倍
  \begin{align*}
    \shazo{\bullet \cdot \bullet}
          {\RR\times \RR^n}{\RR^n}
          {(a,x)}{ax}
  \end{align*}
  は二項演算である.
  これは$\RR$の$\RR^n$への作用である.
\end{example}

\begin{example}
  実ベクトル$x,y$に対し, その内積を対応させる
  \begin{align*}
    \shazo{\Braket{\bullet,\bullet}}
          {\RR^n\times \RR^n}{\RR}
          {(x,y)}{\Braket{x,y}}
  \end{align*}
  も二項演算である.
  これは$\RR^n$上の$\RR$-形式である.
\end{example}

\begin{definition}
  $X$上の二項演算
  \begin{align*}
    \shazo{f}
          {X\times X}{X}
          {(x,y)}{f(x,y)}
  \end{align*}
  について考える.
  $X'\subset X$が次の条件を満たすとき,
  $X'$は$f$に関して閉じているという:
  \begin{enumerate}
  \item $x',y'\in X'\implies f(x',y')\in X'$.
  \end{enumerate}
\end{definition}
\begin{remark}
  $f$を$X$上の二項演算とし,
  $X'\subset X$とする.
  $X'$は$f$について閉じているというのは,
  $f$(の制限)は$X'$上の二項演算でもあるということである,
\end{remark}



\begin{definition}
  $X$の$Y$上の作用
  \begin{align*}
    \shazo{f}
          {X\times Y}{Y}
          {(x,y)}{f(x,y)}
  \end{align*}
  について考える.
  $Y'\subset Y$が次の条件を満たすとき,
  $Y'$は$f$に関して閉じているという:
  \begin{enumerate}
  \item $x\in X$, $y\in Y\implies f(x,y)\in Y'$.
  \end{enumerate}
\end{definition}
\begin{remark}
  $f$を$X$の$Y$上の作用とし,
  $Y'\subset Y$とする.
  $Y'$が$f$について閉じているというのは,
  $f$(の制限)は$X$の$Y'$上の作用でもあるということである,
\end{remark}


\begin{definition}
  $X$上の形式
  \begin{align*}
    \shazo{f}
          {X\times X}{Y}
          {(x,y)}{f(x,y)}
  \end{align*}
  について考える.
  $X'\subset X$に対し,
  その制限,
  \begin{align*}
    \shazo{f|_{X'\times X'}}
          {X'\times X'}{Y}
          {(x,y)}{f(x,y)}
  \end{align*}
  を$f$の$X'$への制限と呼ぶ.
  文脈上問題ない場合には,
  $f$の$X'$への制限も同じ記号$f$を使って表す.
\end{definition}
\begin{remark}
  $f$を$X$上の$Y$-形式とし,
  $X'\subset X$とする.
  $f$の$X'$への制限は,
  $X'$上の$Y$-形式である.
\end{remark}


\section{$X$上の二項演算に関する用語}
ここでは, $X$上の二項関係に関して用いられる用語について
いくつか紹介する.

\subsection{可換律}
ここでは,
$\ast$を$X$上の中置記法の二項演算とする.
つまり
\begin{align*}
  \shazo{\bullet\ast\bullet}
        {X\times X}{X}
        {(x,y)}{x\ast y}
\end{align*}
とする.
\begin{definition}
  $x,y\in X$が
  次の条件を満たすとき,
  $x$と$y$は可換であるという:
  \begin{align*}
    x\ast y=y\ast x.
  \end{align*}
\end{definition}
\begin{definition}
  次の条件を満たすとき,
  $\ast$は可換律を満たすという:
  \begin{align*}
    x,y\in X\implies x\ast y=y\ast x.
  \end{align*}
\end{definition}
\begin{remark}
  $\ast$が可換律を満たすことを,
  $\ast$が可換であるともいう,
\end{remark}
\begin{example}
  $\NN$上の二項演算として,
  通常の和$+$を考える.
  和$+$は可換である.
\end{example}
\begin{example}
  $\NN$上の二項演算として,
  通常の積$\cdot$を考える.
  積$\cdot$は可換である.
\end{example}
\begin{nonexample}
\begin{align*}
  \shazo{\bullet^\bullet}
        {\ZZ_{>0}\times \ZZ_{>0}}{\ZZ_{>0}}
        {(x,y)}{x^y}
\end{align*}
は
$\ZZ_{>0}$上の二項演算である.
\begin{align*}
  2^3&=8,\\
  3^2&=9
\end{align*}
であるので,
この演算は可換ではない.
\end{nonexample}
\begin{nonexample}
  $n$次正方行列同士の積も$n$次正方行列である.
  $n$次正方行列上の二項演算として積を考える.
  \begin{align*}
    \begin{pmatrix}
      0&1\\0&0
    \end{pmatrix}
    \begin{pmatrix}
      0&0\\1&0
    \end{pmatrix}
    &=
    \begin{pmatrix}
      1&0\\0&0
    \end{pmatrix}
    \\
    \begin{pmatrix}
      0&0\\1&0
    \end{pmatrix}
    \begin{pmatrix}
      0&1\\0&0
    \end{pmatrix}
    &=
    \begin{pmatrix}
      0&0\\0&1
    \end{pmatrix}
  \end{align*}
  であるので,
  積は可換ではない.
\end{nonexample}

\begin{nonexample}
  $n$次正方行列$A$, $B$に対し,
  \begin{align*}
    [A,B]=AB-BA
  \end{align*}
  とおくと,
  $[\bullet,\bullet]$は,
  $n$次正方行列上の二項演算である.
  この演算をリー・ブラケットと呼ぶ.
  \begin{align*}
    [A,B]&=AB-BA\\
    [B,A]&=BA-AB=-(AB-BA)
  \end{align*}
  であるので
  $[A,B]=-[B,A]$である.
  \begin{align*}
    E&=\begin{pmatrix}0&1\\0&0\end{pmatrix}\\
    F&=\begin{pmatrix}0&0\\1&0\end{pmatrix}
  \end{align*}
  とおくと,
  \begin{align*}
    [E,F]&=EF-FE=
    \begin{pmatrix}
      1&0\\0&0
    \end{pmatrix}
    -
    \begin{pmatrix}
      0&0\\0&1
    \end{pmatrix}
    =
    \begin{pmatrix}
      1&0\\0&-1
    \end{pmatrix}\\
    [F,E]&=-[E,F]=
    \begin{pmatrix}
      -1&0\\0&1
    \end{pmatrix}
  \end{align*}
  となるので, $[\bullet,\bullet]$は可換ではない.
\end{nonexample}

$\ast$が
可換である
というの演算自身の性質であり,
その演算が閉じた部分集合でも自然に成り立つ.
\begin{prop}
  $\ast$は
  可換
  な
  $X$上の二項演算とし,
  $X'\subset X$は$\ast$に関して閉じているとする.
  このとき,
  $X'$上の二項演算として,
  $\ast$は
  可換
  である.
\end{prop}

$X$上の二項演算について可換であるということを
定義したが,
より一般に$X$上の$Y$-形式についても,
定義できる.

\begin{definition}
  二項演算
  \begin{align*}
    \shazo{\bullet\ast\bullet}
          {X\times X}{Y}
          {(x,y)}{x\ast y}
  \end{align*}
  に対し,
  $x,y\in X$が
  次の条件を満たすとき,
  $x$と$y$は可換であるという:
  \begin{align*}
    x\ast y=y\ast x.
  \end{align*}
\end{definition}
\begin{definition}
  二項演算
  \begin{align*}
    \shazo{\bullet\ast\bullet}
          {X\times X}{Y}
          {(x,y)}{x\ast y}
  \end{align*}
  に対し,
  次の条件を満たすとき,
  $\ast$は可換律を満たすという:
  \begin{align*}
    x,y\in X\implies x\ast y=y\ast x.
  \end{align*}
\end{definition}
\begin{remark}
  $\ast$が可換律を満たすことを,
  $\ast$が可換であるともいう,
  また$\ast$が対称であるともいう,
\end{remark}


\subsection{結合律}
ここでは,
$\ast$を$X$上の中置記法の二項演算とする.
\begin{definition}
  次の条件を満たすとき,
  $\ast$は結合律を満たすという:
  \begin{align*}
    x,y,z\in X\implies (x\ast y)\ast z = x\ast (y\ast z).
  \end{align*}
\end{definition}
\begin{remark}
  $\ast$が結合律を満たすことを,
  $\ast$が結合的であるともいう,
\end{remark}
\begin{example}
  $\NN$上の二項演算として,
  通常の和$+$を考える.
  和$+$は結合律を満たす.
\end{example}
\begin{nonexample}
\begin{align*}
  \shazo{\bullet^\bullet}
        {\ZZ_{>0}\times \ZZ_{>0}}{\ZZ_{>0}}
        {(x,y)}{x^y}
\end{align*}
は
$\ZZ_{>0}$上の二項演算である.
\begin{align*}
  (3^3)^3&=9^3=729,\\
  3^{(3^3)}&=3^9=19683
\end{align*}
であるので,
この演算は結合律を満たさない.
\end{nonexample}

\begin{nonexample}
  $n$次正方行列上の二項演算としてリー・ブラケットを考える.
  \begin{align*}
    [A,A]&=AA-AA=O_{n}
  \end{align*}
  である.  ただし$O_n$は零行列を表す.
  \begin{align*}
    E&=\begin{pmatrix}0&1\\0&0\end{pmatrix}\\
    F&=\begin{pmatrix}0&0\\1&0\end{pmatrix}\\
    H&=\begin{pmatrix}1&0\\0&-1\end{pmatrix}
  \end{align*}
  とおくと,
  \begin{align*}
    [[E,E],F]&=[O_n,F]=O_nF-FO_n=O_n\\
    [E,[E,F]]&=[E,H]=-2E
  \end{align*}
  となるので, リー・ブラケットは結合的ではない.
\end{nonexample}
\begin{nonexample}
  $2$元集合$X=\Set{\top, \bot}$上の演算$\overline{\bullet\lor\bullet}$を
  \begin{align*}
    \overline{\top \lor \top} &= \bot&
    \overline{\top \lor \bot} &= \bot\\
    \overline{\bot \lor \top} &= \bot&
    \overline{\bot \lor \bot} &= \top    
  \end{align*}
  で定める.
  このとき,
  \begin{align*}
    \overline{\bot \lor \overline{\bot \lor \top}} &=\overline{\bot \lor \bot}=\top\\
    \overline{\overline{\bot \lor \bot} \lor \top}&=
    \overline{\top \lor \top}=\bot
  \end{align*}
  であるので, この演算は結合的でない.
\end{nonexample}

\begin{remark}
  $\ast$が結合律が成り立つ演算ならば,
  \begin{align*}
    a\ast b \ast c
  \end{align*}
  を
  \begin{align*}
    (a\ast b) \ast c
  \end{align*}
  と解釈しても,
  \begin{align*}
    a\ast (b \ast c)
  \end{align*}
  と解釈しても等しい元となる.
  これは,
  \begin{align*}
    a_1\ast a_2 \ast \cdots \ast a_n
  \end{align*}
  のような場合でも同様である.
  そのため,
  結合の順序を表すカッコを省略することが多い.
\end{remark}
\begin{remark}
  $\ast$が$X$上の結合的な二項演算であるとき,
  正の整数$n$に対し,
  \begin{align*}
    \underbrace{a\ast \cdots \ast a}_{n}
  \end{align*}
  は$a\in X$と$n$のみから決まる.
  そこで, これを, $a$と$n$のみを使って略記することがある.
  演算によってその略記法は異なるが,
  $a^n$とか$na$などと書かれることがある.
  また複数の演算が扱われているときには
  $a^{\ast n}$などと演算の記号を伴って書かれる記法もある.

  和と呼ばれる演算は
  $na$と略記されることが多く,
  積と呼ばれる演算は,
  $a^n$と略記されることが多いように思う.
\end{remark}

$\ast$が
結合的である
というの演算自身の性質であり,
その演算が閉じた部分集合でも自然に成り立つ.
\begin{prop}
  $\ast$は
  結合的
  な
  $X$上の二項演算とし,
  $X'\subset X$は$\ast$に関して閉じているとする.
  このとき,
  $X'$上の二項演算として,
  $\ast$は
  結合的
  である.
\end{prop}



\subsection{単位元}
ここでは,
$\ast$を$X$上の中置記法の二項演算とする.
\begin{definition}
  $e\in X$とする.
  \begin{enumerate}
  \item
    次の条件を満たすとき,
    $e$は$\ast$に関し左単位的であるという:
    \begin{enumerate}
    \item $x\in X\implies e\ast x=x$.
    \end{enumerate}
  \item
    次の条件を満たすとき,
    $e$は$\ast$に関し右単位的であるという:
    \begin{enumerate}
    \item $x\in X\implies x\ast e=x$.
    \end{enumerate}
  \item
    $e$が,
    $\ast$に関し左単位的かつ
    $\ast$に関し右単位的であるとき,
    $e$は$\ast$に関し単位的であるという.
  \end{enumerate}
\end{definition}
\begin{lemma}
  $e\in X$が左単位的であり,
  $e'\in X$が右単位的であるなら,
  $e=e'$,
\end{lemma}
\begin{proof}
  $e\ast e'$について考える.
  $e$は左単位的であるので, $e\ast e'=e'$.
  $e'$は右単位的であるので, $e\ast e'=e$.
  したがって, $e=e'$.
\end{proof}
\begin{cor}
\label{cor:uniq:unit}
  $e\in X$, $e'\in X$が単位的なら,
  $e=e'$.
\end{cor}
\begin{remark}
  \Cref{cor:uniq:unit}により,
  単位的な元は存在するなら, ただ一つであることがわかる.
  単位的な元を (もし存在するなら)
  $\ast$の単位元と呼ぶ.

  $\ast$が和と呼ばれるときには,
  $\ast$の単位元は零元や加法単位元と呼ばれ
  $0$や$O$にまつわる記号が用いられることが多く,
  $\ast$が積と呼ばれるときには, $\ast$の単位元は乗法単位元と呼ばれ
  $1$や$E$, $I$にまつわる記号が用いられることが多いように思う.
\end{remark}
\begin{cor}
  右単位元と左単位元が存在するなら,
  それらは一致し$\ast$の単位元である.
\end{cor}
\begin{cor}
  $\ast$の単位元が存在するとする.
  \begin{enumerate}
  \item 左単位的な元は, $\ast$の単位元である.
  \item 右単位的な元は, $\ast$の単位元である.
  \end{enumerate}
\end{cor}
\begin{cor}
  $\ast$は可換な二項演算であるとする.
  \begin{enumerate}
  \item 左単位的な元は, $\ast$の単位元である.
  \item 右単位的な元は, $\ast$の単位元である.
  \end{enumerate}
\end{cor}
\begin{example}
  $\ZZ$における通常の和$+$を考える.
  $0$は$+$に関する単位元である.
\end{example}
\begin{example}
  $\ZZ_{>0}$における通常の積$\cdot$を考える.
  $1$は$\cdot$に関する単位元である.
\end{example}

\begin{example}
  二点集合$\Set{0,1}$上の演算$\rhd$
  を
  \begin{align*}
    0\rhd 0 &= 0&
    0\rhd 1 &= 1\\
    1\rhd 0 &= 0&
    1\rhd 1 &= 1
  \end{align*}
  で定義する.
  このとき, $0$は左単位的である.
  また$1$も左単位的である.
\end{example}

\begin{nonexample}
  $\ZZ_{>0}$上の和$+$について考える.
  $x\in\ZZ_{>0}$とする.
  このとき, 
  $x+1>1$となるので$x+1=1$となることはない.
  したがって, $+$の単位元は存在しない.
\end{nonexample}


$\ast$に関して閉じている部分集合を考えるとき,
二項演算が単位元を持つかどうかは,
考えている集合による.
$\ast$が単位元をもっていたとしても,
$\ast$に関して部分集合に単位元が存在しているかはわからない.
しかし,
単位元が満たしている条件自体は演算の性質であり,
$\ast$の単位元が
$\ast$に関して部分集合に含まれているなら,
その元は単位元となる.
\begin{prop}
  $\ast$は
  $X$上の二項演算とし,
  $X'\subset X$は$\ast$に関して閉じているとする.
  $e\in X$は左単位的であるとし,
  $e\in X'$とする.
  このとき,
  $e$は,
  $X'$上の二項演算$\ast$でも,
  左単位的である.
\end{prop}
\begin{prop}
  $\ast$は
  $X$上の二項演算とし,
  $X'\subset X$は$\ast$に関して閉じているとする.
  $e\in X$は右単位的であるとし,
  $e\in X'$とする.
  このとき,
  $e$は,
  $X'$上の二項演算$\ast$でも,
  右単位的である.
\end{prop}
\begin{prop}
  \label{prop:unit:inher}
  $\ast$は
  $X$上の二項演算とし,
  $e\in X$は$\ast$の単位元であるとする.
  $X'\subset X$は$\ast$に関して閉じているとする.
  $e\in X'$ならば,
  $e$は,
  $X'$上の二項演算としての
  $\ast$の単位元である.
\end{prop}
\begin{remark}
  $X$上の二項演算$\ast$が単位元をもつとし,
  その単位元を$e$とする.
  $X'\subset X$は$\ast$に関して閉じているととする.
  $e\in X'$ならば,
  \cref{prop:unit:inher}より,
  $X'$上の二項演算$\ast$も単位元をもち,
  その単位元は$e$である.
  一方, $e\not\in X'$であっても,
  $X'$上の二項演算$\ast$も単位元をもつことはある.
  この場合,
  その単位元を$e'$とすると
  $e\neq e'$であり, $e'$は$X$上の演算$\ast$の単位元ではない.
\end{remark}
\begin{example}
  $2$元集合$X=\Set{\top, \bot}$上の演算$\bullet\lor\bullet$を
  \begin{align*}
   {\top \lor \top} &= \top&
   {\top \lor \bot} &= \top\\
   {\bot \lor \top} &= \top&
   {\bot \lor \bot} &= \bot    
  \end{align*}
  で定める.
  このとき, $\bot$は$\lor$の単位元である.
  $X'=\Set{\bot}$とおくと, $X'$は$\lor$に関して閉じている.
  $\bot \in X'$であり, $\bot$は$X$上の演算$\lor$の単位元であったから,
  $X'$上の演算としての$\lor$の単位元でもある.
  $X''=\Set{\top}$とおくと, $X''$は$\lor$に関して閉じている.
  $\bot\not\in X''$であるが,
  $\top$は$X''$上の演算としての$\lor$の単位元である.
\end{example}

$X$上の二項演算に関して単位的な元というものを定義したが,
これは, $X$の$Y$上の作用でも定義できる.
\begin{definition}
  二項演算
  \begin{align*}
    \shazo{\bullet\ast\bullet}
          {X\times Y}{Y}
          {(x,y)}{x\ast y}
  \end{align*}
  に対し,
  $e\in X$が次の条件を満たすとき,
  $e$は$\ast$に関し左単位的であるという:
  \begin{enumerate}
  \item $x\in X\implies e\ast x=x$.
  \end{enumerate}
\end{definition}

\subsection{逆元}
ここでは, $\ast$を$X$上の中置記法の二項演算とし,
$\ast$が単位元$e$を持つとする.
\begin{definition}
  $e\in X$は$\ast$の単位元であるとする.
  $x\in X$とする.
  \begin{enumerate}
  \item 次の条件を満たす$y$を$x$の左逆元と呼ぶ:
    \begin{enumerate}
      \item $y\ast x=e$.
    \end{enumerate}
  \item 次の条件を満たす$y$を$x$の右逆元と呼ぶ:
    \begin{enumerate}
      \item $x\ast y=e$.
    \end{enumerate}
  \item 次の条件を満たす$y$を$x$の逆元と呼ぶ:
    \begin{enumerate}
      \item $y\ast x=e$.
    \item $x\ast y=e$.
    \end{enumerate}  
  \end{enumerate}
  逆元をもつ$x\in X$を可逆元と呼ぶことがある.
\end{definition}

\begin{lemma}
  $\ast$は結合的であるとし,
  $x\in X$とする.
  $y\in X$が$x$の左逆元であり,
  $y'\in X$が$x$の右逆元であり,
  $y=y'$.
\end{lemma}
\begin{proof}
  $\ast$の単位元を$e$とする.
  $y$は$x$の左単位元であるので,
  $(y\ast x) \ast y'=e\ast y'=y'$.
  $y$は$x$の右単位元であるので,
  $y\ast (x \ast y')=y\ast e=y$.
  $\ast$は結合的なので,
  $(y\ast x) \ast y'=y\ast (x \ast y')$.
  したがって, $y=y'$.
\end{proof}
\begin{cor}
\label{cor:uniq:inv}
  $\ast$は結合的であるとし,
  $x\in X$とする.
  $y\in X$, $y'\in X$が$x$の逆元なら,
  $y=y'$.
\end{cor}
\begin{remark}
  $\ast$が結合的であるとき,
  \Cref{cor:uniq:inv}により,
  $x\in X$の逆元は, 存在するなら, $x\in X$に対して,
  ただ一つであることがわかる.
  そこで, $x\in X$の逆元を$x$を用いて表すことが多い.
  $\ast$が和と呼ばれるときには,
  $x$の逆元は$-x$のような記号が用いられることが
  多く,
  $\ast$が積と呼ばれるときには, 
  $x$の逆元は$x^{-1}$のような記号が用いられることが
  多いように思う.
\end{remark}
\begin{cor}
  $\ast$は結合的であるとする.
  $x\in X$の右逆元と左逆元が存在するなら,
  それらは一致し$x$の逆元である.
\end{cor}
\begin{cor}
  $\ast$は結合的であるとし,
  $x\in X$とする.
  $x$の逆元が存在するとする.
  \begin{enumerate}
  \item $x$の左逆元は, $x$の逆元である.
  \item $x$の右逆元は, $x$の逆元である.
  \end{enumerate}
\end{cor}
\begin{cor}
  $\ast$は結合的かつ可換律を満たす二項演算であるとし,
  $x\in X$とする.
  \begin{enumerate}
  \item $x$の右逆元は, $x$の逆元である.
  \item $x$の右逆元は, $x$の逆元である.
  \end{enumerate}
\end{cor}
\begin{example}
  $\ZZ$上の通常の和を考える.
  このとき, $0$は$+$の単位元である.
  どの$x\in \ZZ$も逆元を持ち,
  $x$の逆元はその$-1$倍の$-x$である.
\end{example}
\begin{example}
  $\QQ$上の通常の積を考える.
  このとき, $1$は$\cdot$の単位元である.
  どの$x\in\QQ$に対しても$0x=0$であり,
  $0x=1$となることはない.
  したがって$0$の逆元はない.
  $0$以外の$x\in \QQ$は逆元を持ち,
  $x$の逆元はその逆数$\frac{1}{x}$である.
\end{example}
\begin{example}
  $\ZZ$上の通常の積を考える.
  このとき, $1$は$\cdot$の単位元である.
  $0\in\ZZ$は逆元を持たない.
  また, $x>1$に対して,
  $0<\frac{1}{x}<1$であるので,
  $\frac{1}{x}\not\in \ZZ$である.
  したがって, $x>1$ならば$x\in\ZZ$は逆元を持たない.
  同様に, $x<-1$ならば$x\in \ZZ$は逆元を持たない.
  $1$の逆元は$1$であり,
  $-1$の逆元は$-1$である.
\end{example}

$\ast$に関して閉じている部分集合を考えるとき
$x$が逆元を持つかどうかは
考えている集合による.
$x^{-1}$がもとの集合に存在していても,
部分集合に存在しているかどうかはわからない.
しかし,
逆元が満たしている条件自体は演算の性質であり,
もとの集合の単位元が含まれており,
逆元が部分集合に含まれているなら,
その元は逆元となる.
\begin{prop}
  $\ast$は
  $X$上の二項演算とし,
  $e\in X$は$\ast$の単位元であるとする.
  $X'\subset X$は$\ast$に関して閉じており,
  $e\in X'$とする.
  $x\in X$の逆元$y\in X$が$y\in X'$をみたすなら,
  $y$は,
  $X'$上の二項演算としての
  $x$の逆元である.
\end{prop}
\begin{prop}
  $\ast$は
  $X$上の二項演算とし,
  $e\in X$は$\ast$の単位元であるとする.
  $X'\subset X$は$\ast$に関して閉じており,
  $e\in X'$とする.
  $x\in X$の左逆元$y\in X$が$y\in X'$をみたすなら,
  $y$は,
  $X'$上の二項演算としての
  $x$の左逆元である.
\end{prop}
\begin{prop}
  $\ast$は
  $X$上の二項演算とし,
  $e\in X$は$\ast$の単位元であるとする.
  $X'\subset X$は$\ast$に関して閉じており,
  $e\in X'$とする.
  $x\in X$の右逆元$y\in X$が$y\in X'$をみたすなら,
  $y$は,
  $X'$上の二項演算としての
  $x$の右逆元である.
\end{prop}


\subsection{分配律}


\chapter{代数系}
\section{いくつかの代数系の定義}
\section{群}
\section{環}
\section{体}
\sectionX{章末問題}
\begin{enumerate}
\item 可換かつ結合的な二項演算の例を挙げよ.
\item 非可換かつ結合的な二項演算の例を挙げよ.
\item 可換かつ非結合的な二項演算の例を挙げよ.
%\item 非可換かつ非結合的な二項演算の例を挙げよ.
\end{enumerate}



    
