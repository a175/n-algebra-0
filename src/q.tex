\section{$X$上の二項演算の続き}
ここでは, $X$上の二項関係に関して用いられる用語について
いくつか紹介する.
以下では,
$\ast$を$X$上の中置記法の二項演算とする.
つまり
\begin{align*}
  \shazo{\bullet\ast\bullet}
        {X\times X}{X}
        {(x,y)}{x\ast y}
\end{align*}
とする.

\subsection{吸収元}
\begin{definition}
  $e\in X$とする.
  \begin{enumerate}
  \item
    次の条件を満たすとき,
    $z$は$\ast$に関し左吸収的であるという:
    \begin{enumerate}
    \item $x\in X\implies z\ast x=z$.
    \end{enumerate}
  \item
    次の条件を満たすとき,
    $e$は$\ast$に関し右吸収的であるという:
    \begin{enumerate}
    \item $x\in X\implies x\ast z=x$.
    \end{enumerate}
  \item
    $z$が,
    $\ast$に関し左吸収的かつ
    $\ast$に関し右吸収的であるとき,
    $z$は$\ast$に関し吸収的であるという.
  \end{enumerate}
\end{definition}

\begin{lemma}
  $z\in X$が左吸収的であり,
  $z'\in X$が右吸収的であるなら,
  $z=z'$,
\end{lemma}
\begin{proof}
  $z\ast z'$について考える.
  $z$は左吸収的であるので, $z\ast z'=z$.
  $z'$は右吸収的であるので, $z\ast z'=z'$.
  したがって, $z=z'$.
\end{proof}
\begin{cor}
\label{cor:uniq:absorbinelm}
  $e\in X$, $e'\in X$が単位的なら,
  $e=e'$.
\end{cor}
\begin{remark}
  \Cref{cor:uniq:absorbinelm}により,
  吸収的な元は存在するなら, ただ一つであることがわかる.
  吸収的な元を (もし存在するなら)
  $\ast$の吸収元(absorbing elment)と呼ぶ.

  代数系によっては, 吸収元のことを零元と呼ぶもあるが,
  加法単位元のことを零元と呼ぶ代数系のほうが多いように思う.
\end{remark}
\begin{cor}
  右吸収元と左吸収元が存在するなら,
  それらは一致し$\ast$の吸収元である.
\end{cor}
\begin{cor}
  $\ast$の吸収元が存在するとする.
  \begin{enumerate}
  \item 左吸収的な元は, $\ast$の吸収元である.
  \item 右吸収的な元は, $\ast$の吸収元である.
  \end{enumerate}
\end{cor}
\begin{cor}
  $\ast$は可換な二項演算であるとする.
  \begin{enumerate}
  \item 左吸収的な元は, $\ast$の吸収元である.
  \item 右吸収的な元は, $\ast$の吸収元である.
  \end{enumerate}
\end{cor}
\begin{example}
  $\ZZ_{\geq 0}$における通常の積$\cdot$を考える.
  $0$は$\cdot$に関する吸収元である.
\end{example}
\begin{example}
  $x,y\in\RR_{\geq 0}$
  に対し, $\min(x,y)$を
  $x$と$y$の小さい方とする.
  このとき, $\min$は
  $\RR_{\geq 0}$上の二項演算である.
  $0$は$\min$に関する吸収元である.
\end{example}

\begin{example}
  二点集合$\Set{0,1}$上の演算$\rhd$
  を
  \begin{align*}
    0\rhd 0 &= 0&
    0\rhd 1 &= 1\\
    1\rhd 0 &= 0&
    1\rhd 1 &= 1
  \end{align*}
  で定義する.
  このとき, $0$は右吸収的である.
  また$1$も右吸収的である.
\end{example}


\begin{nonexample}
  $\ZZ_{>0}$上の積$\cdot$について考える.
  $x\in\ZZ_{>0}$とする.
  このとき, $2\cdot x>x$であるので,
  $2\cdot x=x$となることはない.
  したがって, $\cdot$の吸収元は存在しない.
\end{nonexample}


\subsection{冪等元}
